%Puls-Doppler-Radar

\subsection{Puls-Doppler-Radar}

Beim Puls-Doppler-Radar werden durch die Antenne des Radars kurze Sendepulse in dieselbe Richtung emittiert. Auf jeden Sendepuls folgt eine gewissen Empfangszeit, in der die Antenne auf Empfang geschaltet wird. Ein Vorteil des Puls-Doppler-Radars besteht also in der Möglichkeit, eine einzige Antenne sowohl für das Senden als auch das Empfangen verwenden zu können. 

Vergleicht man einen Sendepuls mit dessen Echo, so stellt man bei bewegten Objekten einen Unterschied im Frequenzverlauf fest. Dieses Phänomen wird als \glqq Dopplereffekt\grqq bezeichnet. Durch den Vergleich von mehreren Echos verschiedener Sendepulse lassen sich Rückschlüsse auf die Geschwindigkeit (genauer die Radialgeschwindigkeit bezogen auf das Radar) von detektierten Objekten ziehen. 